%  SingleXBs
%  Created by Dave Williams on 2009-06-25.
%  Copyright (c) 2009. All rights reserved.

\documentclass[]{article}

\usepackage{setspace, float, fancyhdr}
\usepackage[pdftex]{graphicx}
\usepackage[utf8]{inputenc}
\usepackage[round,numbers,sort&compress]{natbib} 
\addtolength{\parskip}{\baselineskip}
\setlength{\parindent}{0in}

% Multipart figures
%\usepackage{subfigure}
% Package for including code in the document
%\usepackage{listings}

\title{Multidimensional Crossbridges: Not a rope of sand}
\author{C Dave, Mikey R, Tommy D}
\date{2009-06-25}

\begin{document}

\maketitle

 
\begin{abstract}
	Lorem ipsum dolor sit amet, consectetur adipiscing elit. Fusce id quam et odio viverra fermentum. Donec tincidunt faucibus justo id ultricies. Pellentesque quis quam risus, nec sollicitudin nibh. 
\end{abstract}

Keywords: myosin; spatially-explicit model; crossbridge kinetics

Author Summary: 
Models of muscle contraction have long treated the molecular motor myosin as a simple spring oriented parallel to its direction of movement. 
This does not allow for the investigation of phenomena such as the perpendicular force observed during shortening, or the dependence of the maximum force produced on spacing between the contractile filaments that comprise muscle.
We demonstrate an alternative model, computationally simple enough to use in large networked models, that incorporate both linear and torsional or angular springs. These models capture much of the behavior missing from pervious efforts.

\section{Introduction}
Sarcomere-scale modeling of muscle contraction has largely changed since the introduction of the sliding crossbridge model in the 1950s, but the geometry of the individual crossbridges used has remained largely unaltered. 
While thermodynamic account had been introduced to the crossbridge kinetics, compliance has been introduced to the filaments, and multiple filaments have been arranged to mimic the lattice, the one dimensional single spring nature of the crossbridge has continued to be used as a model of the mechanism of force generation.

\paragraph*{History of Models}

\paragraph*{Increasing knowledge of myosin}

\subparagraph*{X-Ray Diffraction Studies}

\subparagraph*{Single Molecule Experiments}



\section*{Materials and Methods}


\subsection*{Historical 1-D Crossbridge}

The one dimensional crossbridge has remained the dominant model of the crossbridge since its introduction by \citet{Huxley1957e}. The structure of the one dimensional crossbridge has remained largely unchanged while the kinetics underlying transitions between force generating states have been increased in complexity throughout subsequent work. \citep{PateCooke1988, Daniel1998a,Chase2004a,Tanner2007a}
Figures describing the free energy and transition rates of the single spring crossbridge, which are suitable for comparison to Figures \ref{fig:4s} and \ref{fig:4s}, are available from \citet{Tanner2007a}.

The geometry of these one-dimensional models of the crossbridge have been devised with an eye towards accounting for the offset generated by the powerstroke and the energy utilized in the creation of this offset. 
From these two concerns the force the spring (assuming it to be a linear one) can produce is accounted for. 
In a two dimensional model there is the additional goal of replicating the crossbridge's sensitivity to lattice spacing and multidimensional force generated. 

\paragraph*{Geometry}


\begin{eqnarray}
\label{1sEnergy}
    U_1(r) & = & 0 \nonumber \\
    U_2(r) & = & \frac{1}{2}k_r (r-r_s)^2 \nonumber \\
    U_3(r) & = & \frac{1}{2}k_r (r)^2 \\
\end{eqnarray}

 
\paragraph*{Kinetics}

\begin{eqnarray}  
\label{1sTransRates}
	r_{12}(r)   & = & A \sqrt{\frac{k_r}{2 \pi}} e^{-\frac{1}{2} k_r (r-r_s)^2} \nonumber \\
	            & = & A \sqrt{\frac{k_r}{2 \pi}} e^{-U_2(r)} \nonumber \\
    r_{23}(r)   & = & \frac{B}{\sqrt{k_r}} (1 - \tanh(C \sqrt{k_r} (r-r_s))) + D \\
                & = & \frac{B}{\sqrt{k_r}} (1 - \tanh(C \sqrt{2 U_2(r)})) + D \\
	r_{31}(r)   & = & \sqrt{k_r} (\sqrt{M r^2} - N r) + P \\
	r_{12}(r)   & = & \textrm{Dependent on diffusion} \nonumber \\
    r_{23}(r)   & = & 0.001 + 0.5 * (1 + \tanh( \nonumber \\
                        &   & 0.6 (U_1(r, \theta) - U_2(r, \theta)))) \\
	r_{31}(r)   & = & e^{-1 / U_2(r, \theta)}
\end{eqnarray} 
 
Where: $A = 2000$, $B = 100$, $C = 1$, $D = 1$, $M = 3600$, $N = 40$, $P = 20$, and $k_r = 5$ pN/nm



\subsection*{A 2-D 2-Spring Crossbridge}


\paragraph*{Geometry}

The geometry of the two dimensional crossbridge necessitates the introduction of a torsional spring, with the goal of replicating the rotation about the converter domain that produces the powerstroke's offset.


\begin{eqnarray}
\label{2sEnergy}
	U_1(r,\theta) & = & 0 \nonumber \\
    U_2(r,\theta) & = & \frac{1}{2}k_r (r - r_0)^2 + 
                        \frac{1}{2}k_\theta (\theta - \theta_0)^2 \nonumber \\
    U_3(r,\theta) & = & \frac{1}{2}k_r (r - r_1)^2 + 
                        \frac{1}{2}k_\theta (\theta - \theta_1)^2 \\
\end{eqnarray}


\paragraph*{Kinetics}

The binding of both the two and four spring crossbridges is determined by Monte-Carlo simulation of their diffusion as a result of being perturbed by Boltzmann derived energies. 

\begin{eqnarray}  
\label{2sTransRates}
	r_{12}(r, \theta)   & = & \textrm{Dependent on diffusion} \nonumber \\
    r_{23}(r, \theta)   & = & 0.001 + 0.5 * (1 + \tanh( \nonumber \\
                        &   & 0.6 (U_1(r, \theta) - U_2(r, \theta)))) \\
	r_{31}(r, \theta)   & = & e^{-1 / U_2(r, \theta)}
\end{eqnarray} 

\paragraph*{Cutting out cross-sections}



\subsection*{A 2-D 4-Spring Crossbridge}


\paragraph*{Geometry and Force/Displacement Equations}


\begin{eqnarray}
	\lefteqn{U(\phi,\ell,r,\theta) = }  \nonumber \\
 	& & \frac{1}{2}k_\phi (\phi-\phi_0)^2 + \frac{1}{2}k_\ell (\ell-\ell_0)^2 + \nonumber \\
	& & \frac{1}{2}k_r (r-r_0)^2 + \frac{1}{2}k_\theta (\theta-\theta_0)^2
\end{eqnarray}

\paragraph*{Kinetics: Equations and Figures}
\paragraph*{Cutting out cross-sections}



\section*{Results}

\subsection*{Lattice Spacing Dependence of Binding Rates}



\subsection*{Forward Biased Binding}


\section*{Discussion}

\subsection*{}


\section*{Figures}

\begin{figure}[p]
    \begin{center}
    \includegraphics[width=3.2in]{../imgs/xb_types_and_cycle.pdf}
    \label{fig:types}
    \caption{
        Kinetic scheme and crossbridge types under investigation. 
        A) Three state kinetics used in most models and extended here. 
        B) Single spring crossbridge model used in models since \cite{Huxley1957e}. 
        C) Two spring system, consisting of a torsional/angular spring and a linear spring. 
        D) Four spring system using two torsional and two linear springs.}
    \end{center}
\end{figure}

\begin{figure}[p]
    \begin{center}
    \includegraphics[width=3.2in]{../imgs/2s_contours_and_cuts.pdf}
    \label{fig:2s}
    \caption{
        Energy and kinetics of the two spring crossbridge at varying horizontal offsets from a resting location and varying lattice spacings.}
    \end{center}
\end{figure}

\begin{figure}[p]
    \begin{center}
    %\includegraphics[width=3.2in]{../imgs/4s_contours_and_cuts.pdf}
    \label{fig:4s}
    \caption{
        Energy and kinetics of the four spring crossbridge at varying horizontal offsets from a resting location and varying lattice spacings.}
    \end{center}
\end{figure}

\begin{figure}[p]
    \begin{center}
    %\includegraphics[width=3.2in]{../imgs/force_vectors.pdf}
    \label{fig:force}
    \caption{
        Force vectors of the two and four spring crossbridges, and the differences between.}
    \end{center}
\end{figure}



% Bib style requires biophysj.bst be in the document directory
\bibliographystyle{biophysj}
\bibliography{SingleXB}

\end{document}
