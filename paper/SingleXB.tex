%
%  SingleXB
%
%  Created by Dave Williams on 2009-06-25.
%  Copyright (c) 2009. All rights reserved.
%
\documentclass[]{article}

\usepackage[pdftex]{graphicx}
\usepackage[utf8]{inputenc}
\usepackage{fancyhdr}


% Multipart figures
%\usepackage{subfigure}
% Surround parts of graphics with box
%\usepackage{boxedminipage}
% Package for including code in the document
%\usepackage{listings}



\title{Multidimensional Crossbridges: Not a rope of sand}
\author{C Dave, Mikey R, Tommy D}
\date{2009-06-25}

\begin{document}

\maketitle


\begin{abstract}
	Lorem ipsum dolor sit amet, consectetur adipiscing elit. Fusce id quam et odio viverra fermentum. Donec tincidunt faucibus justo id ultricies. Pellentesque quis quam risus, nec sollicitudin nibh. 

Keywords: myosin; spatially-explicit model; crossbridge kinetics}} \\[.5em]

Author Summary: 
	Models of muscle contraction have long treated the molecular motor myosin as a simple spring oriented parallel to its direction of movement. This does not allow for the investigation of phenomena such as the perpendicular force observed during shortening, or the dependence of the maximum force produced on spacing between the contractile filaments that comprise muscle. We demonstrate an alternative model, computationally simple enough to use in large networked models, that incorporate both linear and torsional or angular springs. These models capture much of the behavior missing from pervious efforts.
\end{abstract}

\section{Introduction}

\bibliographystyle{plain}
\bibliography{}
\end{document}
